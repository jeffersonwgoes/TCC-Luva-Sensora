\chapter{Conclusões}

	O presente trabalho resultou em aprimoramento em um sistema de reconhecimento de letras do alfabeto de LIBRAS, permitindo detectar letras que contém movimento e completando as letras identificáveis pelo sistema. O código foi portado, obviamente com alterações para funcionar corretamente e a coleta dos dados do sensor inercial e as redes neurais funcionaram. A única coisa que não foi possível cumprir foi o teste com outros usuários para validar o funcionamento do sistema.


    A luva, a qual já havia sido construída em um trabalho anterior, possui fios e sensores expostos e frágeis. Então o manuseio da luva teve que ser bem delicado pois os fios das bobinas sempre soltavam, necessitando reparos. Uma solução pra isso em um trabalho futuro seria trocar a luva e esconder a fiação. Também escolher uma luva mais flexível, pois em alguns sinais é necessário fazer um pouco de força para a sinalização correta da letra, o que também pode vir a danificar os sensores. A comunicação I2C utilizada com o sensor inercial se mostrou pouco robusta por causa do tamanho do cabo (inicialmente 60 cm), que teve que ser reduzido para não causar tantos problemas de comunicação.
    
    Com relação ao microcontrolador, este foi suficiente para a demanda de processamento. Apesar disso, devido a abordagem de \textit{software} com \textit{background/foreground}, foi necessário temporizar e diminuir a frequência que se capturava os sinais. Como a aplicação é sensível ao tempo, seria interessante aplicar um \textit{protothread} ou um sistema operacional de tempo real para eliminar algumas esperas ocupadas.
    
    Para trabalhos futuros, como identificar sinais de LIBRAS simbolizam palavras e utilizam apenas uma mão e sem expressões faciais, seria necessário uma analise sobre o sensor inercial, trocando algumas posições de sensores para as características da rede neural ficarem mais marcantes. Para o processamento dos dados do sensor inercial poderia ser estudado e implementado o código que o microcontrolador do sensor já realiza os cálculos dos ângulos de rotação(\textit{Pitch, Yaw}) e se aplicar um sensor inercial de 9 ou 10 graus de liberdade para melhorar a precisão dos sinais lidos do sensor inercial.

    Caso seja a intenção de implementar sinais com duas mãos, seria necessário outra luva, mas a demanda de tempo para a leitura das duas poderia ser elevada. Por isso deve-se utilizar a abordagem da luva com mais tipos de sensores, por exemplo alguns sensores resistivos de flexão, que são lidos mais rapidamente e simples de interpretar, juntamente com sensores inerciais de 9 ou 10 graus de liberdade.
    
    E  por fim, para implementar um sistema completo para identificar sinais e expressões faciais, seria necessário implementar processamento de imagem.
    
